\documentclass[11pt]{article}
\usepackage[utf8]{inputenc}
\usepackage[a4paper,margin=1in]{geometry}
\usepackage{amsmath,amssymb,mathtools}
\usepackage{enumitem}
\setlist[enumerate]{itemsep=0.3em, topsep=0.3em}
\setlist[itemize]{itemsep=0.2em, topsep=0.2em}
\newcommand{\ds}{\displaystyle}
\newtheorem{claim}{Claim}

\begin{document}

\begin{center}
{\Large \textbf{Math 55 — HW 9}}\\[6pt]
\end{center}

\vspace{0.75em}
\noindent \textbf{Due on Gradescope Wednesday, Oct. 29, 8pm}

\bigskip
\noindent \textbf{Problems from Rosen}

\medskip
\noindent \textbf{Section 5.3}
\begin{itemize}
    \item[12.] Prove that $f_1^2 + f_2^2 + \ldots + f_n^2 = f_n f_{n+1}$ when $n$ is a positive integer.

    \medskip\textit{Answer:}
    We prove by induction on $n$. For $n=1$, $f_1^2=1=f_1 f_2$ since $f_1=f_2=1$. Assume the identity holds for $n=k$:
    \[
      \sum_{i=1}^k f_i^2 = f_k f_{k+1}.
    \]
    Then
    \[
      \sum_{i=1}^{k+1} f_i^2
      = \Big(\sum_{i=1}^k f_i^2\Big)+f_{k+1}^2
      = f_k f_{k+1} + f_{k+1}^2
      = f_{k+1}(f_k+f_{k+1})
      = f_{k+1} f_{k+2},
    \]
    using $f_{k+2}=f_{k+1}+f_k$. Thus the statement holds for $k+1$. By induction it holds for all $n\ge1$.

    \item[14.] Show that $f_{n+1} f_{n-1} - f_n^{2} = (-1)^n$ when $n$ is a positive integer.

    \medskip\textit{Answer:}
    Use Cassini’s identity. For $n=1$: $f_2 f_0 - f_1^2 = 1\cdot0-1=-1=(-1)^1$.
    Assume $f_{k+1}f_{k-1}-f_k^2=(-1)^k$ for some $k\ge1$. Then
    \[
    \begin{aligned}
      f_{k+2}f_k - f_{k+1}^2
        &= (f_{k+1}+f_k)f_k - f_{k+1}^2 \\
        &= -\big(f_{k+1}^2 - f_{k+1}f_k - f_k^2\big) \\
        &= -\big(f_{k+1}(f_{k+1}-f_k) - f_k^2\big) \\
        &= -\big(f_{k+1}f_{k-1} - f_k^2\big) \\
        &= -(-1)^k = (-1)^{k+1},
    \end{aligned}
    \]
    where we used $f_{k-1}=f_{k+1}-f_k$. This completes the induction.
\end{itemize}

\noindent \textbf{Section 6.1}
\begin{itemize}
    \item[18.] How many 5-element DNA sequences
    \begin{enumerate}[label=(\alph*)]
        \item end with A?
        \item start with T and end with G?
        \item contain only A and T?
        \item do not contain C?
    \end{enumerate}

    \medskip\textit{Answer:}
    The DNA alphabet is $\{A,C, G, T\}$ (4 symbols).
    \begin{enumerate}[label=(\alph*)]
        \item Fix last symbol to A (1 choice); the first four positions are arbitrary (4 choices each): $4^4=256$.
        \item Fix first to T and last to G (1 choice each); remaining three positions arbitrary: $4^3=64$.
        \item Only A or T allowed in all five positions: $2^5=32$.
        \item Exclude C; allowed symbols $\{A,G,T\}$ in each of five positions: $3^5=243$.
    \end{enumerate}

    \item[24.] How many positive integers between 1000 and 9999 inclusive
    \begin{enumerate}[label=(\alph*)]
        \setcounter{enumi}{2}
        \item have distinct digits?
        \setcounter{enumi}{4}
        \item are divisible by 5 or 7?
        \item are not divisible by either 5 or 7?
        \item are divisible by 5 but not by 7?
        \item are divisible by 5 and 7?
    \end{enumerate}

    \medskip\textit{Answer:}
    There are $9000$ four-digit integers in total (from 1000 to 9999).
    \begin{enumerate}[label=(\alph*)]
        \setcounter{enumi}{2}
        \item Distinct digits: $9$ choices for the thousands (1--9), then $9$ for hundreds (0--9 except the thousands digit), then $8$, then $7$. Total $9\cdot9\cdot8\cdot7=4536$.
        \setcounter{enumi}{4}
        \item By inclusion–exclusion: multiples of $5$: $\lfloor 9999/5 \rfloor-\lfloor 999/5 \rfloor = 1999-199=1800$. Multiples of $7$: $\lfloor 9999/7 \rfloor-\lfloor 999/7 \rfloor = 1428-142=1286$. Multiples of $35$: $\lfloor 9999/35 \rfloor-\lfloor 999/35 \rfloor = 285-28=257$. Hence $1800+1286-257=2829$.
        \item Not divisible by $5$ nor $7$: $9000-2829=6171$.
        \item Divisible by $5$ but not by $7$: $1800-257=1543$.
        \item Divisible by both $5$ and $7$ (i.e., by $35$): $257$.
    \end{enumerate}

    \item[48.] In how many ways can a photographer at a wedding arrange 6 people in a row from a group of 10 people, where the bride and the groom are among these 10 people, if
    \begin{enumerate}[label=(\alph*)]
        \item the bride must be in the picture?
        \item both the bride and groom must be in the picture?
        \item exactly one of the bride and the groom is in the picture?
    \end{enumerate}

    \medskip\textit{Answer:}
    Each selected set of 6 distinct people can be ordered in $6!$ ways.
    \begin{enumerate}[label=(\alph*)]
        \item Include the bride and choose the other $5$ from the remaining $9$: $\binom{9}{5}6! = 126\cdot720 = 90{,}720$.
        \item Include both bride and groom and choose the other $4$ from the remaining $8$: $\binom{8}{4}6! = 70\cdot720 = 50{,}400$.
        \item Choose exactly one of the two (2 ways), then choose the other $5$ from the $8$ non-spouses: $2\binom{8}{5}6! = 2\cdot56\cdot720 = 80{,}640$.
    \end{enumerate}

    \item[76.] Use mathematical induction to prove the product rule for $m$ tasks from the product rule for two tasks.

    \medskip\textit{Answer:}
    Our claim: If task $i$ can be performed in $n_i$ ways for $i=1,2,\dots,m$, and the tasks are independent (choices for one do not affect others), then the number of ways to perform all $m$ tasks in sequence is $\prod_{i=1}^m n_i$.

    \emph{Base $m=2$.} Given as the standard product rule: $n_1 n_2$.

    \emph{Inductive step.} Assume true for some $m=k$: any $k$ tasks can be completed in $\prod_{i=1}^k n_i$ ways. For $k+1$ tasks, first complete the initial $k$ tasks (by the IH in $\prod_{i=1}^k n_i$ ways). For each such outcome there are $n_{k+1}$ ways to complete task $k+1$. Hence total $\big(\prod_{i=1}^k n_i\big) n_{k+1} = \prod_{i=1}^{k+1} n_i$. This establishes the rule for all $m\ge2$.
\end{itemize}

\pagebreak
\noindent \textbf{Additional Problems}
\begin{enumerate}
    \item What is wrong with the following proof that for all $n \ge 3$, we can form postage of $n$ cents using only 3 and 4 cent stamps?

    \medskip
    \noindent\emph{“Proof:”} For $n = 3$, we can use one 3 cent stamp. For $n > 3$, assume by induction that we can form postage of $n-1$ cents. Then we can form postage of $n$ cents by replacing one 3 cent stamp with a 4 cent stamp, or by replacing two 4 cent stamps with three 3 cent stamps.

    \medskip\emph{Answer: }
    The induction hypothesis does not guarantee that the $n-1$-cent configuration contains a 3-cent stamp (needed to “replace by a 4”), nor two 4-cent stamps (needed to “replace by three 3s”). For example, $n-1=6$ can be made as $3+3$ (no 4-cent stamp to replace), and $n-1=8$ can be $4+4$ (no 3-cent stamp). Thus the step is invalid: it assumes extra structure not ensured by the hypothesis. A correct proof would use strong induction showing that any amount $\ge 6$ can be formed, handling small bases (3,4,6,7) and then adding 3 or 4 appropriately.

    \item For any set $S$ and any function $f: S \to S$, define $f^{(0)}(x)=x$ and $f^{(n)}(x)=f(f^{(n-1)}(x))$ for $n>0$.
    \begin{enumerate}[label=(\alph*)]
        \item For $f(x)=x^2$ on $\mathbb{R}$, find a formula for $f^{(n)}(x)$ and prove it.

        \emph{Answer: } We claim $f^{(n)}(x)=x^{2^n}$. Proof by induction on $n$: Base $n=0$: $x^{2^0}=x=f^{(0)}(x)$. Assume true for $n=k$. Then $f^{(k+1)}(x)=f(f^{(k)}(x))=(x^{2^k})^2=x^{2^{k+1}}$.

        \item Prove that for any $m,n\ge0$, $f^{(m+n)}(x)=f^{(m)}(f^{(n)}(x))$.

        \emph{Answer: } Fix $n$ and prove by induction on $m$. Base $m=0$: $f^{(n)}(x)=f^{(0)}(f^{(n)}(x))$. Assume $f^{(m+n)}=f^{(m)}\circ f^{(n)}$. Then $f^{(m+1+n)}=f(f^{(m+n)})=f\big(f^{(m)}(f^{(n)}(x))\big)=(f^{(m+1)}\circ f^{(n)})(x)$. Thus the identity holds for all $m,n$.
    \end{enumerate}
\end{enumerate}

\end{document}
