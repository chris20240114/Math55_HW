\documentclass{article}
\usepackage[utf8]{inputenc}
\usepackage{amsmath, amssymb}

\begin{document}

\begin{center}
    \textbf{\LARGE Math 55—Fall 2025—Haiman} \\[6pt]
    \textbf{\Large Homework 7} \\[6pt]
\end{center}

\vspace{1em}

\bigskip

Some of the problems on this homework set require quite a bit of computation using a calculator (a programmable calculator may be helpful) or a computer. For that reason, I’ve assigned fewer problems than usual.

\bigskip

\noindent \textbf{Problems from Rosen}

\medskip

\noindent \textbf{Section 4.6}
\begin{itemize}
    \item[24.] Show your steps.

    \item[26.] Modified as follows. Use the same $n = 53 \cdot 61$ as in the problem, but use $e = 253$ for the encryption exponent, and $2642\;3218\;2887\;1068$ for the encrypted message. (It takes less work to find the answer with these numbers than with the numbers in the book.)

    \item[28.]

    \item[30.] Describe the steps and also find their shared key!

    \item[32.] (Following the instructions preceding Exercises 31–33.)
\end{itemize}

\bigskip

\noindent \textbf{Additional problems:}

\begin{enumerate}
    \item Pollard’s factoring algorithm is described in the supplementary notes on factoring on bCourses (see the list of lecture topics and reading). Use Pollard’s algorithm to factor the number $n = 7081$. Show your steps.

    \item[(2a)] Show that if $a^2 \equiv b^2 \pmod{n}$, where $a \not\equiv \pm b \pmod{n}$, then $\gcd(a+b,n)$ and $\gcd(a-b,n)$ are non-trivial factors of $n$ (i.e., they are factors and they are not equal to 1 or to $n$).

    \item[(2b)] Use part (a) to factor the number $n = 6893$ by trying the first several integers $a$ larger than $\sqrt{n}$ until you find one for which $a^2 \bmod n$ is a perfect square.

    \medskip
    \textit{FYI:} The method in this problem is Fermat’s factoring algorithm. If $n$ is composite, it typically takes about $\sqrt[4]{n}$ steps to find a number $a$ such that $a^2 \bmod n$ is a perfect square and factor $n$, similar to the performance of Pollard’s algorithm.
\end{enumerate}

\end{document}
