\documentclass{article}
\usepackage[utf8]{inputenc}
\usepackage{amsmath, amssymb}
\usepackage{geometry}
\geometry{a4paper, margin=1in}

\begin{document}

\begin{center}
    \textbf{\large Stat 134 Fall 2025: Homework 5} \\[6pt]
    Shobhana Stoyanov \\[6pt]
    \small Due: October 10
\end{center}

\vspace{1em}

\noindent Please turn in the following problems on Gradescope by FRIDAY, October 10, 11:59 PM. You may handwrite your answers and scan them, or type them up with \LaTeX, whichever you prefer. Make sure to submit a single pdf and assign your problems to pages. Your work should be legible, and will be graded on completion, so it should be easy to see if you have completed the problems. You will be tested on these materials in the quiz, and the homework problems as well as the other problems covered in class and discussion are intended to help you practice the concepts that you learn in lecture and the text. Each problem is marked out of 1 point for completion, and to get this point, you must show your reasoning.

\bigskip

\noindent \textbf{Collaboration, using AI or other resources:} You are encouraged to work with your colleagues, especially at the homework parties. Talking through the problems as you think about them is invaluable, but then each person should write down their solution themselves. Please indicate on your homework who you worked with and any other resources you might have used, including GenAI tools. Note that we are not policing your use of AI, but it is better to try the problems yourself first, as you won't have the crutch of AI on quizzes and exams. It is worth putting in the time now to build the muscle you will need to complete harder problems. If you don't start exercising your problem solving skills now, you will struggle with later material.

\bigskip

\subsection*{Random variables, PMFs, Independence}

\begin{enumerate}
    \item[\textbf{1.}] [\#23 from chapter 3] There are $n$ people eligible to vote in a certain election. Voting requires registration. Decisions are made independently. Each of the $n$ people will register with probability $p_{1}$. Given that a person registers, they will vote with probability $p_{2}$. Given that a person votes, they will vote for Kodos (who is one of the candidates) with probability $p_{3}$. What is the distribution of the number of votes for Kodos (give the PMF, fully simplified, or the name of the distribution, including its parameters)?

    \item[\textbf{2.}] [\#24 from chapter 3] Let $X$ be the number of Heads in 10 fair coin tosses.
    \begin{enumerate}
        \item[(a)] Find the conditional PMF of $X$, given that the first two tosses both land Heads. \\
        (Hint: you need to find compute probabilities of the type $P(X=k \mid \text{First 2 tosses land Heads})$. What are the possible values of k? What do these probabilities look like?)
        \item[(b)] Find the conditional PMF of $X$, given that at least two tosses land Heads. \\
        (Hint: What is the definition of $P(X=k|X\ge2)$?)
    \end{enumerate}

    \item[\textbf{3.}] [\#33 from chapter 3] A book has $n$ typos. Two proofreaders, Prue and Frida, independently read the book. Prue catches each typo with probability $p_{1}$ and misses it with probability $q_{1}=1-p_{1}$, independently, and likewise for Frida, who has probabilities $p_{2}$ of catching and $q_{2}=1-p_{2}$ of missing each typo. Let $X_{1}$ be the number of typos caught by Prue, $X_{2}$ be the number caught by Frida, and $X$ be the number caught by at least one of the two proofreaders.
    \begin{enumerate}
        \item[(a)] Write down the distribution of $X_{1}$, $X_{2}$, and $X$.
        \item[(b)] What is the expected value of $X$?
        \item[(c)] Assume $p_{1}=p_{2}$. Use Bayes' Rule to write down the conditional distribution of $X_{1}$ given that $X_{1}+X_{2}=t$.
    \end{enumerate}
    
    \item[\textbf{4.}] [\#38 from chapter 3]
    \begin{enumerate}
        \item[(a)] Give an example of dependent r.v.s $X$ and $Y$ such that $P(X<Y)=1$.
        \item[(b)] Give an example of independent r.v.s $X$ and $Y$ such that $P(X<Y)=1$.
    \end{enumerate}
\end{enumerate}

\subsection*{Expected values}

\begin{enumerate}
    \setcounter{enumi}{4} % Start numbering from 5
    \item[\textbf{5.}] [\#1 from chapter 4] Bobo, the amoeba from Chapter 2, currently lives alone in a pond. After one minute Bobo will either die, split into two amoebas, or stay the same, with equal probability. Find the expectation and variance for the number of amoebas in the pond after one minute.

    \item[\textbf{6.}] [\#3 from chapter 4]
    \begin{enumerate}
        \item[(a)] A fair die is rolled. Find the expected value of the roll.
        \item[(b)] Four fair dice are rolled. Find the expected total of the rolls.
    \end{enumerate}

    \item[\textbf{7.}] Consider an event $A$. We can define the indicator random variable $I_{A}$ to be 1 if $A$ is true, and 0 otherwise. We have seen that the expected value of $I_{A}=P(A)$. Now suppose we have a list of events $A_{1},A_{2},...,A_{n}$ such that $X$ counts the number of events that occur from among this list, so $X = I_{1}+I_{2}+...+I_{n}$. Use this idea to compute $E(X)$ in the following problems:
    \begin{enumerate}
        \item[(a)] Let $X$ be the number of aces in a 5-card poker hand.
        \item[(b)] A system has $n$ components, and at any particular time, the $k$th component is working with probability $p_{k}$, for $k=1,2,...,n$. Let $X$ be the number of components working at that time.
        \item[(c)] A company that makes dog treats. They tested 12 different flavors of treats on 10 dogs, all of whom picked one flavor out of the 12 at random, independently of all the other dogs. Let $X$ be the number of flavors picked by at least one dog.
    \end{enumerate}

    \item[\textbf{8.}] Prove the tail sum formula for expectation: For any $X$ with values in $\{0,1,2,...,n\}$ we have that $E(X)=\sum_{j=1}^{n}P(X\ge j)$. \\
    (Hint: try using indicators, what will your $A_{j}$ be?)
\end{enumerate}

\end{document}
