\documentclass{article}
\usepackage[utf8]{inputenc}
\usepackage{amsmath, amssymb}
\usepackage{geometry}
\geometry{a4paper, margin=1in}

\begin{document}

\begin{center}
    \textbf{\LARGE Math 55—Fall 2025—Haiman} \\[6pt]
    \textbf{\Large Homework 5} \\[6pt]
\end{center}

\noindent \textbf{Problems from Rosen}

\medskip

\noindent \textbf{Section 4.3}
\begin{itemize}
    \item[42.] Use the extended Euclidean algorithm to express $\gcd(252,356)$ as a linear combination of $252$ and $356$.
    \medskip\\
    \textit{Answer: First, we apply the Euclidean algorithm to find the gcd:
    \[
    \begin{alignedat}{2}
        356&=1\cdot252+104\\
        252&=2\cdot104+44\\
        104&=2\cdot44+16\\
        44&=2\cdot16+12\\
        16&=1\cdot12+4\\
        12&=3\cdot4+0
    \end{alignedat}
    \]
    The last non-zero remainder is $4$, so $\gcd(252, 356) = 4$.
    \medskip\\
    Next, we use back-substitution to express $4$ as a linear combination of $252$ and $356$:
    \[
    \begin{aligned}
        4&=16-1\cdot12\\
         &=16-1\cdot(44-2\cdot16) &&\text{Substitute } 12 = 44 - 2 \cdot 16 \\
         &=16 - 44 + 2\cdot16 = 3\cdot16-44\\
         &=3(104-2\cdot44)-44 &&\text{Substitute } 16 = 104 - 2 \cdot 44 \\
         &=3\cdot104-6\cdot44-44 = 3\cdot104-7\cdot44\\
         &=3\cdot104-7(252-2\cdot104) &&\text{Substitute } 44 = 252 - 2 \cdot 104 \\
         &=3\cdot104-7\cdot252+14\cdot104 = 17\cdot104-7\cdot252\\
         &=17(356-1\cdot252)-7\cdot252 &&\text{Substitute } 104 = 356 - 1 \cdot 252 \\
         &=17\cdot356-17\cdot252-7\cdot252 = 17\cdot356-24\cdot252.
    \end{aligned}
    \]
    Hence, $\gcd(252,356)=4= (-24)\cdot 252 + 17\cdot 356$.}

    \medskip

    \item[46(e).] Find the smallest positive integer with exactly $10$ different positive factors.
    \medskip\\
    \textit{Answer: The number of positive divisors of an integer $n$ with prime factorization $n=p_1^{a_1}p_2^{a_2}\cdots p_k^{a_k}$ is given by $\tau(n)=(a_1+1)(a_2+1)\cdots(a_k+1)$. We need $\tau(n)=10$. The ways to factor $10$ are $10$ and $5 \cdot 2$. This leads to two cases for the exponents:
    \begin{itemize}
        \item \textbf{Case 1:} $a_1+1 = 10 \implies a_1=9$. To minimize $n=p_1^9$, we choose the smallest prime, $p_1=2$. This gives $n=2^9=512$.
        \item \textbf{Case 2:} $(a_1+1)(a_2+1)=5 \cdot 2 \implies a_1=4$ and $a_2=1$. To minimize $n=p_1^4 p_2^1$, we assign the larger exponent to the smaller prime. Thus, we choose $p_1=2$ and $p_2=3$, which gives $n=2^4\cdot 3^1=16\cdot 3=48$.
    \end{itemize}
    Comparing the two cases, $48 < 512$. The smallest such integer is $\boxed{48}$.}

    \medskip

    \item[54.] Adapt Euclid’s proof to show there are infinitely many primes of the form $3k+2$.
    \medskip\\
    \textit{Answer: We prove this by contradiction.
    \begin{enumerate}
        \item Assume there are only a finite number of primes of the form $3k+2$. Let this finite list of primes be $p_1, p_2, \dots, p_r$.
        \item Let $N = 3(p_1 p_2 \cdots p_r) - 1$.
        \item We consider the prime factorization of $N$. Notice that $N \equiv -1 \equiv 2 \pmod{3}$. This implies that $3$ is not a prime factor of $N$.
        \item Furthermore, for any prime $p_i$ in our list, $N \equiv -1 \pmod{p_i}$. This means that none of the primes $p_i$ divides $N$.
        \item Since $N>1$, it must have at least one prime factor. All prime factors of $N$ must be of the form $3k+1$ or $3k+2$. They cannot all be of the form $3k+1$, because the product of numbers of the form $3k+1$ is itself of the form $3k+1$. (For example, $(3k_1+1)(3k_2+1) = 3(3k_1k_2+k_1+k_2)+1 \equiv 1 \pmod 3$.)
        \item Since $N \equiv 2 \pmod 3$, at least one of its prime factors, let's call it $q$, must be of the form $3k+2$.
        \item This prime $q$ is of the form $3k+2$, but $q$ cannot be in our original list $\{p_1, \dots, p_r\}$, since none of those primes divide $N$. This contradicts our assumption that our list contained all primes of the form $3k+2$.
    \end{enumerate}
    Thus, the assumption must be false, and there are infinitely many primes of the form $3k+2$.}
\end{itemize}

\bigskip

\noindent \textbf{Section 4.4}
\begin{itemize}
    \item[6(b).] Find an inverse of $a$ modulo $m$ for $a=34,\; m=89$.
    \medskip\\
    \textit{Answer: We use the extended Euclidean algorithm. First, the division steps:
    \[
    \begin{alignedat}{2}
        89&=2\cdot 34+21 \\
        34&=1\cdot 21+13 \\
        21&=1\cdot 13+8 \\
        13&=1\cdot 8+5 \\
        8&=1\cdot 5+3 \\
        5&=1\cdot 3+2 \\
        3&=1\cdot 2+1
    \end{alignedat}
    \]
    Now, we perform back-substitution to write $1$ as a linear combination of $34$ and $89$:
    \[
    \begin{aligned}
        1 &= 3 - 1 \cdot 2 \\
        &= 3 - 1 \cdot (5 - 1 \cdot 3) = 2 \cdot 3 - 1 \cdot 5 \\
        &= 2 \cdot (8 - 1 \cdot 5) - 1 \cdot 5 = 2 \cdot 8 - 3 \cdot 5 \\
        &= 2 \cdot 8 - 3 \cdot (13 - 1 \cdot 8) = 5 \cdot 8 - 3 \cdot 13 \\
        &= 5 \cdot (21 - 1 \cdot 13) - 3 \cdot 13 = 5 \cdot 21 - 8 \cdot 13 \\
        &= 5 \cdot 21 - 8 \cdot (34 - 1 \cdot 21) = 13 \cdot 21 - 8 \cdot 34 \\
        &= 13 \cdot (89 - 2 \cdot 34) - 8 \cdot 34 = 13 \cdot 89 - 26 \cdot 34 - 8 \cdot 34 \\
        &= 13 \cdot 89 - 34 \cdot 34.
    \end{aligned}
    \]
    From $1 = 13 \cdot 89 - 34 \cdot 34$, we see that $1 \equiv (-34) \cdot 34 \pmod{89}$.
    So, the inverse of $34$ is $-34 \equiv -34 + 89 \equiv \boxed{55} \pmod{89}$.}

    \medskip

    \item[12(a).] Solve the congruence $34x \equiv 77 \pmod{89}$ using the inverse from 6(b).
    \medskip\\
    \textit{Answer: From 6(b), the inverse of $34$ modulo $89$ is $55$. We multiply both sides of the congruence by $55$:
    \[
        x \equiv 55 \cdot 77 \pmod{89}.
    \]
    To simplify the calculation, note that $77 \equiv -12 \pmod{89}$.
    \[
        x \equiv 55 \cdot (-12) \equiv -660 \pmod{89}.
    \]
    To find the value of $-660 \pmod{89}$, we can add multiples of $89$:
    $-660 + 8 \cdot 89 = -660 + 712 = 52$.
    So, $x \equiv \boxed{52} \pmod{89}$.}

    \medskip

    \item[16.]
    \begin{itemize}
        \item[(a)] Show that the positive integers less than $11$, except $1$ and $10$, can be split into pairs of integers that are inverses of each other modulo $11$.
        \medskip\\
        \textit{Answer: We need to find pairs $(a,b)$ from $\{2,3,4,5,6,7,8,9\}$ such that $ab \equiv 1 \pmod{11}$.
        The pairs are:
        \[
        (2,6) \text{ since } 2\cdot6=12 \equiv 1 \pmod{11},
        \]
        \[
        (3,4) \text{ since } 3\cdot4=12 \equiv 1 \pmod{11},
        \]
        \[
        (5,9) \text{ since } 5\cdot9=45=4\cdot11+1 \equiv 1 \pmod{11},
        \]
        \[
        (7,8) \text{ since } 7\cdot8=56=5\cdot11+1 \equiv 1 \pmod{11}.
        \]
        These four pairs include all integers from $2$ to $9$.}

        \medskip

        \item[(b)] Use part (a) to show that $10! \equiv -1 \pmod{11}$.
        \medskip\\
        \textit{Answer: We can write out $10!$ and group the pairs of inverses:
        \[
        \begin{aligned}
            10! &= 1 \cdot 2 \cdot 3 \cdot 4 \cdot 5 \cdot 6 \cdot 7 \cdot 8 \cdot 9 \cdot 10 \\
            &\equiv 1 \cdot (2\cdot6) \cdot (3\cdot4) \cdot (5\cdot9) \cdot (7\cdot8) \cdot 10 \pmod{11} \\
            &\equiv 1 \cdot (1) \cdot (1) \cdot (1) \cdot (1) \cdot 10 \pmod{11} \\
            &\equiv 10 \pmod{11} \\
            &\equiv \boxed{-1} \pmod{11}.
        \end{aligned}
        \]}
    \end{itemize}
\end{itemize}

\newpage

\noindent \textbf{Additional problems}

\begin{itemize}
    \item[1.] By Bézout’s theorem, given any $a,b\in \mathbb{Z}^+$, there are integers $r,s$ such that $\gcd(a,b)=ra+sb$. Prove that if $(r_0, s_0)$ is one such pair, then any other pair $(r,s)$ must be of the form $(r,s) = \left(r_0 - k\frac{b}{g}, s_0 + k\frac{a}{g}\right)$ for some integer $k$, where $g=\gcd(a,b)$.
    \medskip\\
    \textit{Answer: Let $g=\gcd(a,b)$, and suppose $(r_0,s_0)$ is a particular solution to $r_0a+s_0b=g$. Let $(r,s)$ be any other solution, so $ra+sb=g$.
    Subtracting the first equation from the second gives:
    \[
        (r-r_0)a + (s-s_0)b = 0 \implies (r-r_0)a = -(s-s_0)b.
    \]
    Let $a=ga_0$ and $b=gb_0$, where $\gcd(a_0,b_0)=1$. Substituting these into the equation gives:
    \[
        (r-r_0)ga_0 = -(s-s_0)gb_0 \implies (r-r_0)a_0 = -(s-s_0)b_0.
    \]
    This shows that $a_0$ divides the product $(s-s_0)b_0$. Since $\gcd(a_0,b_0)=1$, by Euclid's Lemma, we must have $a_0 \mid (s-s_0)$. Therefore, $s-s_0 = ka_0$ for some integer $k$.
    Substituting this back into the equation $(r-r_0)a_0 = -(s-s_0)b_0$:
    \[
        (r-r_0)a_0 = -(ka_0)b_0 \implies r-r_0 = -kb_0.
    \]
    So, we have found that any other solution $(r,s)$ must satisfy:
    \[
        s = s_0 + ka_0 = s_0 + k\frac{a}{g} \quad \text{and} \quad r = r_0 - kb_0 = r_0 - k\frac{b}{g}.
    \]
    Conversely, any pair of this form is a valid solution, because:
    \[
    \Big(r_0-k\frac{b}{g}\Big)a+\Big(s_0+k\frac{a}{g}\Big)b = (r_0a+s_0b) - k\frac{ab}{g} + k\frac{ab}{g} = g.
    \]
    Thus, all solutions are of the specified form. \hfill$\square$}

    \medskip

    \item[2.] Prove that for all $a,b\in \mathbb{Z}^+$, the set of common divisors of $a$ and $b$ equals the set of divisors of $\gcd(a,b)$.
    \medskip\\
    \textit{Answer: We need to prove that for any integer $d$, ($d\mid a$ and $d\mid b$) $\iff$ $d\mid \gcd(a,b)$.
    \begin{itemize}
        \item[($\Rightarrow$)] Let $d$ be a common divisor of $a$ and $b$. Let $g=\gcd(a,b)$. By Bézout's theorem, there exist integers $r,s$ such that $g=ra+sb$. Since $d\mid a$ and $d\mid b$, $d$ must divide any linear combination of $a$ and $b$. Thus, $d \mid (ra+sb)$, which means $d\mid g$.
        \item[($\Leftarrow$)] Let $d$ be a divisor of $g=\gcd(a,b)$. By the definition of the greatest common divisor, we know that $g\mid a$ and $g\mid b$. Since $d\mid g$ and $g\mid a$, by the transitivity of divisibility, we have $d\mid a$. Similarly, since $d\mid g$ and $g\mid b$, we have $d\mid b$. Thus $d$ is a common divisor of $a$ and $b$.
    \end{itemize}
    This proves the two sets of divisors are identical. \hfill$\square$}

    \medskip

    \item[3.] Solve linear congruences $ax\equiv b \pmod m$ when $a$ and $m$ are not necessarily coprime.
    \begin{itemize}
        \item[(a)] Let $g=\gcd(a,m)$. Show that if $g\nmid b$, then $ax\equiv b \pmod m$ has no solution.
        \medskip\\
        \textit{Answer: The congruence $ax\equiv b\pmod m$ is equivalent to the equation $ax-b = km$ for some integer $k$. This can be rewritten as $ax - km = b$.
        Let $g=\gcd(a,m)$. By definition, $g \mid a$ and $g \mid m$. Therefore, $g$ must divide any integer linear combination of $a$ and $m$, which includes $ax-km$.
        So, $g \mid b$. This shows that if a solution exists, it is necessary that $g$ divides $b$.
        Therefore, by contraposition, if $g\nmid b$, then the congruence has no solution. \hfill$\square$}

        \medskip

        \item[(b)] If $g\mid b$, set $a'=a/g$, $b'=b/g$, $m'=m/g$. Show $\gcd(a',m')=1$ and that $ax\equiv b\pmod m$ has the same solutions as $a'x\equiv b'\pmod{m'}$.
        \medskip\\
        \textit{Answer:
        First, we show $\gcd(a',m')=1$. Since $g=\gcd(a,m)$, we can write $a=ga'$ and $m=gm'$. Suppose $d=\gcd(a',m')$. Then $d \mid a'$ and $d \mid m'$, which implies that $dg \mid ga'$ and $dg \mid gm'$. Thus, $dg$ is a common divisor of $a$ and $m$. Because $g$ is the \emph{greatest} common divisor, we must have $dg \le g$. Since $g>0$, this implies $d\le 1$. As a gcd, $d$ must be positive, so $d=1$.
        \medskip\\
        Next, we show the congruences are equivalent.
        The congruence $ax \equiv b \pmod m$ is equivalent to the equation $ax-b = km$ for some integer $k$.
        Since $g \mid a$, $g \mid b$, and $g \mid m$, we can divide the entire equation by $g$:
        $$ \frac{a}{g}x - \frac{b}{g} = k \frac{m}{g} \iff a'x - b' = k m' $$
        This new equation is precisely the definition of the congruence $a'x \equiv b' \pmod{m'}$. Since the step of dividing or multiplying by the non-zero integer $g$ is reversible, the two congruences have the exact same set of integer solutions for $x$. \hfill$\square$}

        \medskip

        \item[(c)] Use (b) to solve $28x\equiv 12 \pmod{40}$ and list all solutions modulo $40$.
        \medskip\\
        \textit{Answer: Here, $a=28$, $b=12$, and $m=40$.
        First, we find $g=\gcd(28,40)=4$.
        Since $g=4$ and $4\mid 12$, solutions exist. We reduce the congruence using the results from part (b):
        \[ a' = 28/4=7, \quad b'=12/4=3, \quad m'=40/4=10. \]
        The equivalent congruence is $7x\equiv 3 \pmod{10}$.
        To solve this, we find the inverse of $7$ modulo $10$. By inspection, $7 \cdot 3 = 21 \equiv 1 \pmod{10}$, so $7^{-1} \equiv 3 \pmod{10}$.
        Multiplying the congruence by $3$:
        \[ x \equiv 3 \cdot 3 \equiv 9 \pmod{10}. \]
        This means the solutions are of the form $x=9+10k$ for any integer $k$. We want the solutions in the range $[0, 39]$.
        \begin{itemize}
            \item $k=0: x = 9$
            \item $k=1: x = 19$
            \item $k=2: x = 29$
            \item $k=3: x = 39$
        \end{itemize}
        There are $g=4$ incongruent solutions modulo $40$. The set of solutions is $\boxed{\{9, 19, 29, 39\}}$.}
    \end{itemize}
\end{itemize}

\end{document}